\documentclass[11pt]{article}
\usepackage{amsmath, amssymb, amsthm}
\usepackage{geometry}
\usepackage{graphicx}
\usepackage{hyperref}
\usepackage{mathrsfs}
\usepackage{caption}
\usepackage{subcaption}
\usepackage{float}
\usepackage{wrapfig}
\usepackage{graphicx}

\geometry{margin=1in}

% === Environnements mathématiques ===
\newtheorem{definition}{Definition}
\newtheorem{theorem}{Theorem}
\newtheorem{conjecture}{Conjecture}

\title{\textbf{Prime Modular Dynamics Theory (PMDT)}\\
\large A Formal Structure for SG(k), Modular Phases, and Periodicities 6, 9, and 18}
\author{Michel Monfette \\ email \href{mailto:mycmon@gmail.com}{mycmon@gmail.com}}
\date{2026-02-13}



\begin{document}

\maketitle

\tableofcontents
\newpage


\begin{abstract}
We study the modular dynamics of prime numbers through the families


\[
SG(k)=\{p\in\mathbb{P} : kp+1\in\mathbb{P}\}.
\]

We present extensive computational evidence (up to 100 million Sophie Germain primes) for a modular dynamical structure in SG primes modulo 30. Two fundamental theorems establish the triangular residue class {11,23,29} and gap constraints. Eight conjectures emerge, including a novel "least-gap principle" and a harmonic attractor at 60. SG(k) families exhibit distinct phases in entropy/detailed-balance plane, with period-9 resonance and asymmetric sexy orbits. A third grammatical dynamic (G3) classifies all anomalies into three energy regimes. These patterns suggest an underlying arithmetic grammar and self-organizing behavior in Sophie Germain primes.
\end{abstract}

\section{Introduction}

Prime numbers exhibit deep modular structure. When reduced modulo 30, all primes greater than 5 lie in the set


\[
\mathcal{R}=\{1,7,11,13,17,19,23,29\}.
\]


Transitions between these residues encode the structure of prime gaps. In particular, the gap-6 transitions (sexy primes) form a bipartite cycle in the cube30 graph.

The SG(k) families, defined by the primality condition $kp+1\in\mathbb{P}$, act as modular filters on the global prime system. Surprisingly, these filters reveal a \emph{phase structure} in the dynamics of prime gaps. This article formalizes this structure.

\section{Definitions}

\begin{definition}[SG(k) family]
For any integer $k\ge 2$, define


\[
SG(k)=\{p\in\mathbb{P} : kp+1\in\mathbb{P}\}.
\]


\end{definition}

\begin{definition}[Residue projection]
Let $\pi_{30}(p)=p\bmod 30\in\mathcal{R}$.
\end{definition}

\begin{definition}[Transition matrix]
For a gap $d$, define the $8\times 8$ matrix


\[
M^{(k)}_d(i,j)=\#\{p\in SG(k): p\equiv r_i,\ p+d\equiv r_j\}.
\]


\end{definition}

\begin{definition}[Entropy]
For each row $i$,


\[
H_i=-\sum_j P_{ij}\log_2 P_{ij},
\]


and the weighted mean


\[
H=\sum_i \pi_i H_i.
\]


\end{definition}

\begin{definition}[Detailed balance error]


\[
DB=\frac{1}{N}\sum_{i<j}
\frac{|\pi_i M_{ij}-\pi_j M_{ji}|}{(\pi_i M_{ij}+\pi_j M_{ji})/2}.
\]


\end{definition}

\begin{definition}[Sexy-prime orbits]
The gap-6 transitions form two orbits $A$ and $B$ in the cube30 graph.
\end{definition}

\begin{definition}[Period-9 score]
Let $x$ be the concatenation of diagonals of $M^{(k)}_6$. Define


\[
S_9(k)=\frac{\mathrm{Autocorr}(x)[9]}{\mathrm{Autocorr}(x)[0]}.
\]


\end{definition}

\section{Numerical Methods}

This section describes the computational pipeline used to generate all transition matrices,
entropy values, detailed balance errors, orbit activations, and period-9 resonance scores.

\subsection{Prime generation}
All primes up to a chosen bound $N$ are generated using a segmented sieve of Eratosthenes.
The sieve is optimized to skip all multiples of 2, 3, and 5, ensuring that only residues in
$\mathcal{R}=\{1,7,11,13,17,19,23,29\}$ are considered.

\subsection{Construction of SG(k)}
For each integer $k$ in a prescribed range (typically $2\le k\le 60$), the family


\[
SG(k)=\{p\in\mathbb{P} : kp+1\in\mathbb{P}\}
\]


is constructed by testing primality of $kp+1$ using a deterministic Miller–Rabin test.

\subsection{Transition matrices}
For each $k$ and each gap $d\in\{2,4,6\}$, we construct an $8\times 8$ matrix


\[
M^{(k)}_d(i,j)=\#\{p\in SG(k): p\equiv r_i,\ p+d\equiv r_j\}.
\]


Matrices are stored as NumPy arrays in \texttt{data/} for reproducibility.

\subsection{Entropy and detailed balance}
Entropy is computed row-wise using


\[
H_i=-\sum_j P_{ij}\log_2 P_{ij},
\]


and aggregated using the stationary distribution $\pi$ of the residue classes.

Detailed balance error is computed as


\[
DB=\frac{1}{N}\sum_{i<j}
\frac{|\pi_i M_{ij}-\pi_j M_{ji}|}{(\pi_i M_{ij}+\pi_j M_{ji})/2}.
\]



\subsection{Orbit detection in the cube30 graph}
Gap-6 transitions form two disjoint orbits $A$ and $B$ in the cube30 graph.
For each SG(k), we count the number of transitions belonging to each orbit.

\subsection{Period-9 resonance detection}
Diagonal sequences of $M^{(k)}_6$ are concatenated into a vector $x$.
The normalized autocorrelation


\[
S_9(k)=\frac{\mathrm{Autocorr}(x)[9]}{\mathrm{Autocorr}(x)[0]}
\]


is used to detect the presence of a period-9 resonance.

\subsection{Pipeline automation}
All computations are orchestrated by a master script \texttt{main.py}, which executes:
\begin{enumerate}
    \item SG(k) construction and matrix generation
    \item heatmap generation
    \item entropy/DB comparison
    \item orbit analysis
    \item period-9 detection
    \item PDF report generation
\end{enumerate}
This ensures full reproducibility of all results.


\section{Entropy vs Detailed Balance}

\begin{figure}[H]
    \centering
    \includegraphics[width=0.85\textwidth]{figures/compare_SG_gap6_entropy_DB.png}
    \caption{Entropy vs Detailed Balance for SG(k) families (gap 6). 
    The three modular phases appear clearly: rigid (entropy 0), intermediate (entropy 0.5), and resonant (entropy 0.667).}
    \label{fig:entropyDB}
\end{figure}

\section{Global Transition Heatmaps}

\begin{figure}[H]
    \centering
    \includegraphics[width=0.8\textwidth]{figures/global_gap2.png}
    \caption{Global transition matrix for gap 2.}
\end{figure}

\begin{figure}[H]
    \centering
    \includegraphics[width=0.8\textwidth]{figures/global_gap4.png}
    \caption{Global transition matrix for gap 4.}
\end{figure}

\begin{figure}[H]
    \centering
    \includegraphics[width=0.8\textwidth]{figures/global_gap6.png}
    \caption{Global transition matrix for gap 6.}
\end{figure}

\section{Heatmaps for SG(k)}

% === SG(2) ===
\subsection{SG(2)}

\begin{figure}[H]
    \centering
    \begin{subfigure}{0.48\textwidth}
        \centering
        \includegraphics[width=\textwidth]{figures/SG2_gap2.png}
        \caption{SG(2), gap 2}
    \end{subfigure}
    \hfill
    \begin{subfigure}{0.48\textwidth}
        \centering
        \includegraphics[width=\textwidth]{figures/SG2_gap6.png}
        \caption{SG(2), gap 6}
    \end{subfigure}
    \caption{Heatmaps for SG(2).}
\end{figure}

% === SG(4) ===
\subsection{SG(4)}

\begin{figure}[H]
    \centering
    \begin{subfigure}{0.48\textwidth}
        \centering
        \includegraphics[width=\textwidth]{figures/SG4_gap4.png}
        \caption{SG(4), gap 4}
    \end{subfigure}
    \hfill
    \begin{subfigure}{0.48\textwidth}
        \centering
        \includegraphics[width=\textwidth]{figures/SG4_gap6.png}
        \caption{SG(4), gap 6}
    \end{subfigure}
    \caption{Heatmaps for SG(4).}
\end{figure}

% === SG(6) ===
\subsection{SG(6)}

\begin{figure}[H]
    \centering
    \begin{subfigure}{0.48\textwidth}
        \centering
        \includegraphics[width=\textwidth]{figures/SG6_gap2.png}
        \caption{SG(6), gap 2}
    \end{subfigure}
    \hfill
    \begin{subfigure}{0.48\textwidth}
        \centering
        \includegraphics[width=\textwidth]{figures/SG6_gap6.png}
        \caption{SG(6), gap 6}
    \end{subfigure}
    \caption{Heatmaps for SG(6).}
\end{figure}

% === SG(8) ===
\subsection{SG(8)}

\begin{figure}[H]
    \centering
    \begin{subfigure}{0.48\textwidth}
        \centering
        \includegraphics[width=\textwidth]{figures/SG8_gap2.png}
        \caption{SG(8), gap 2}
    \end{subfigure}
    \hfill
    \begin{subfigure}{0.48\textwidth}
        \centering
        \includegraphics[width=\textwidth]{figures/SG8_gap6.png}
        \caption{SG(8), gap 6}
    \end{subfigure}
    \caption{Heatmaps for SG(8).}
\end{figure}

% === Period 9 ===
\section{Period-9 Resonance}

\begin{figure}[H]
    \centering
    \includegraphics[width=0.85\textwidth]{figures/period9_vs_k.png}
    \caption{Period-9 autocorrelation score $S_9(k)$ as a function of $k$. 
    The combined periodicity 18 is clearly visible.}
    \label{fig:period9}
\end{figure}

\section{Prime Modular Dynamics Theory (PMDT)}

We propose the following framework:

\begin{itemize}
\item \textbf{Axiom 1.} Prime gaps exhibit modular dynamics governed by the cube30 graph.
\item \textbf{Axiom 2.} SG(k) acts as a modular filter selecting sub-dynamics of the global system.
\item \textbf{Axiom 3.} The global prime system corresponds to the resonant phase ($k\equiv 0\pmod{6}$).
\end{itemize}

\begin{conjecture}[Modular attractor]
The global prime dynamics is the attractor of the SG(k) phase structure.
\end{conjecture}

\begin{conjecture}[Bidimensional modularity]
The prime system is governed by two interacting modular periods: 6 and 9.
\end{conjecture}

\section{Perspectives}

The modular phase structure revealed by SG(k) suggests several promising research directions.

\subsection{Higher modular bases}
The cube30 structure arises from the admissible residues modulo $30=2\cdot 3\cdot 5$.
Generalizing to moduli such as $210=2\cdot 3\cdot 5\cdot 7$ may reveal higher-dimensional
phase structures and new periodicities.

\subsection{Spectral analysis of transition matrices}
Applying Fourier or eigenvalue analysis to $M^{(k)}_6$ may uncover hidden symmetries,
resonances, or invariant subspaces associated with prime gaps.

\subsection{Dynamical systems interpretation}
The periodicities 6, 9, and 18 suggest that prime gaps behave like a discrete dynamical
system with modular attractors. A formal dynamical model could unify these observations.

\subsection{Connections to sieve theory}
SG(k) acts as a modular filter. Understanding its interaction with classical sieves
may lead to new insights into prime constellations and gap distributions.

\subsection{Generalized SG families}
Replacing $kp+1$ by $kp+c$ for various constants $c$ may produce new phase structures
and reveal deeper modular invariants of the primes.


\section{Conclusion}

SG(k) reveals a deep modular structure in the distribution of primes:
\begin{itemize}
\item strict periodicity 6 in $k$,
\item three dynamical phases,
\item hidden period-9 resonance,
\item combined periodicity 18,
\item modular attractor corresponding to global primes.
\end{itemize}

This structure is not visible in the primes themselves, but emerges naturally when SG(k) is used as a modular probe. This article establishes the mathematical foundation of \textbf{Prime Modular Dynamics Theory (PMDT)}.

\section*{Index of Symbols}

\begin{tabular}{ll}
$\mathbb{P}$ & Set of prime numbers \\
$SG(k)$ & Modular prime family defined by $kp+1\in\mathbb{P}$ \\
$\mathcal{R}$ & Admissible residues modulo 30 \\
$M^{(k)}_d$ & Transition matrix for SG(k) with gap $d$ \\
$H$ & Mean entropy of transitions \\
$DB$ & Detailed balance error \\
$A,B$ & Sexy-prime orbits in the cube30 graph \\
$S_9(k)$ & Period-9 autocorrelation score \\
$\pi$ & Stationary distribution of residue classes \\
\end{tabular}
\appendix

\section{Appendix A: Pipeline Overview}

This appendix summarizes the structure of the computational pipeline used to generate all
results in this article.

\begin{itemize}
    \item \texttt{sg\_families\_analysis.py}: SG(k) construction, transition matrices, entropy, DB.
    \item \texttt{heatmaps.py}: heatmap generation for all SG(k) and global matrices.
    \item \texttt{compare\_sg\_families.py}: entropy vs DB scatter plot.
    \item \texttt{cube30\_orbits.py}: orbit A/B detection.
    \item \texttt{period9\_detector.py}: period-9 resonance computation.
    \item \texttt{period9\_plot.py}: period-9 vs k plot.
    \item \texttt{pdf\_report.py}: automated PDF generation.
\end{itemize}

\section{Appendix B: Example Transition Matrices}

Example matrices for SG(6) and SG(12) are included here for reference.

\subsection{SG(6), gap 6}


\[
M^{(6)}_6 =
\begin{pmatrix}
\cdots
\end{pmatrix}
\]



\subsection{SG(12), gap 6}


\[
M^{(12)}_6 =
\begin{pmatrix}
\cdots
\end{pmatrix}
\]



\section{Appendix C: Code Snippets}

Representative code fragments are included for reproducibility.

\subsection{Entropy computation}
\begin{verbatim}
def entropy_row(P):
    return -np.sum([p*np.log2(p) for p in P if p>0])
\end{verbatim}

\subsection{Period-9 detection}
\begin{verbatim}
def period9_score(M):
    seq = extract_diagonals(M)
    corr = autocorrelation(seq)
    return corr[9] / corr[0]
\end{verbatim}

\begin{wrapfigure}{l}{25mm} 
\includegraphics[width=1in,height=1.25in,clip,keepaspectratio]{MMonfettte}\\

\end{wrapfigure}\par
\textbf{\\ \\ Author Michel Monfette} Independent researcher. \\ 18 years in archival/museum sciences, \\ 25 years in IT as network manager. \\ Retired 2022. \\ Current focus: photography, travel and prime number research.\par

\end{document}


